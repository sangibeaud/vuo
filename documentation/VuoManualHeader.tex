\usepackage[margin=3cm]{geometry}
\usepackage{ucs}
\usepackage{url}
\usepackage{calc}
\usepackage{booktabs}
\usepackage{mparhack}

\textwidth 13cm
\setlength{\parskip}{3ex plus 2ex minus 2ex}
\setlength{\marginparwidth}{4cm}

\usepackage{fancyhdr}
	\pagestyle{fancy}
	\fancyhf{}
	\fancyhead[L]{\color{base1}\bfseries\nouppercase\leftmark}
	\usepackage{lastpage}
	\fancyhead[R]{\color{base1}Page \thepage\ of \pageref*{LastPage}}
	\fancyhfoffset[L]{3em}
	\renewcommand{\headrule}{\vskip -2mm \color{base1}\rule{\headwidth}{0.5pt}}
	\fancyfoot[L]{\color{base1}Revised \today}
	\renewcommand{\footrule}{\vskip 10mm \color{base1}\rule{\headwidth}{0.5pt}\\\vskip -10mm}
	\setlength{\headheight}{14.8pt}

\usepackage{titlesec}
	\newcommand{\sectionbreak}{\clearpage}
	\titleformat{\section}
		{\color{blue}\normalfont\huge\bfseries}
		{\hspace*{-1cm}\thesection}
		{5mm}
		{}
	\titleformat{\subsection}
		{\color{blue}\normalfont\Large\bfseries}
		{\thesubsection}
		{1em}
		{}
	\titleformat{\subsubsection}
		{\color{blue}\normalfont\large\bfseries}
		{\thesubsubsection}
		{1em}
		{}

\usepackage{setspace}
	\onehalfspacing

\usepackage{tikz}
	\usetikzlibrary{shapes,snakes}
	\tikzstyle{sidebox} = [
		inner sep=15pt,
		inner ysep=20pt,
	]
	\tikzstyle{sideboxtitle} = [
		fill=base1,
		rectangle,
		rounded corners,
		text=base3,
		inner sep=5pt,
	]

\newcommand{\noteTextProgrammers}[1]{
	\marginpar{
		\begin{tikzpicture}
			\node[sidebox](box){\begin{minipage}{4.1cm}\linespread{1}\small #1\end{minipage}};
			\node[sideboxtitle, right=10pt] at (box.north west) {\raisebox{-2.5mm}{\includegraphics[height=7mm]{PROJECT_SOURCE_DIR/documentation/image/fa-code.pdf}}\hspace{2mm}\parbox{29.5mm}{\linespread{1}\selectfont Note for\\text programmers}};
		\end{tikzpicture}%
	}
}

\newcommand{\tip}[1]{
	\marginpar{
		\begin{tikzpicture}
			\node[sidebox](box){\begin{minipage}{4.1cm}\linespread{1}\small #1\end{minipage}};
			\node[sideboxtitle, right=10pt] at (box.north west) {\raisebox{-2.5mm}{\hspace{2mm}\includegraphics[height=7mm]{PROJECT_SOURCE_DIR/documentation/image/fa-info.pdf}}\hspace{4mm}\parbox{31.5mm}{Tip}};
		\end{tikzpicture}%
	}
}

\newcommand{\newInVersion}[1]{
	\marginpar{
		\vskip 0pt  % Move the top of the title closer to the command location.
		\begin{tikzpicture}
			\node[sidebox](box){\begin{minipage}{4.1cm}\end{minipage}}; % To maintain consistent spacing with the other sidebars.
			\node[sideboxtitle, right=10pt] at (box.north west) {\raisebox{-2.5mm}{\includegraphics[height=7mm]{PROJECT_SOURCE_DIR/documentation/image/fa-gift.pdf}}\hspace{2mm}\parbox{31.5mm}{New in Vuo #1}};
		\end{tikzpicture}%
	}
}

\newcommand{\changedInVersion}[2]{
	\marginpar{
		\begin{tikzpicture}
			\node[sidebox](box){\begin{minipage}{4.1cm}\linespread{1}\small #2\end{minipage}};
			\node[sideboxtitle, right=10pt] at (box.north west) {\raisebox{-2.5mm}{\hspace{1mm}\includegraphics[height=7mm]{PROJECT_SOURCE_DIR/documentation/image/fa-highlighter.pdf}}\hspace{3mm}\parbox{31.5mm}{Changed in Vuo #1}};
		\end{tikzpicture}%
	}
}

\usepackage[multidot]{grffile}
\newcommand{\doublequote}{\XeTeXglyph\XeTeXcharglyph"0022\relax}
\newcommand{\code}[1]{{\tt #1}}



% =====================================================================
% Glossary

% Workaround for "Package mathspec Error: 'amsmath' must be loaded earlier than `mathspec'."
% https://tex.stackexchange.com/questions/85696/what-causes-this-strange-interaction-between-glossaries-and-amsmath/85700#85700
\makeatletter % undo the wrong changes made by mathspec
\let\RequirePackage\original@RequirePackage
\let\usepackage\RequirePackage
\makeatother

\usepackage[nogroupskip,nopostdot,toc,xindy]{glossaries}
\makeglossaries
\usepackage[xindy]{imakeidx}
\makeindex

% Make term references green and bold.
\renewcommand*{\glstextformat}[1]{\color{green}\bf{#1}}

% Add more space between the definition and the page numbers.
\renewcommand*{\glossentry}[2]{%
  \item[\glsentryitem{#1}%
        \glstarget{#1}{\glossentryname{#1}}]
     \glossentrydesc{#1}\glspostdescription \hspace{1em} #2}%

\newcommand{\term}[1]{\gls{#1}}
\newcommand{\Term}[1]{\Gls{#1}}
\newcommand{\definition}[2]{\newglossaryentry{#1}{name=#1,description={#2}}}
% =====================================================================



\newcommand{\vuoPort}[1]{{\bf #1}}
\newcommand{\vuoNode}[1]{{\bf #1}}
\newcommand{\vuoNodeClass}[1]{{\tt #1}}
\newcommand{\vuoNodeClassImage}[1]{\includegraphics[scale=0.75]{../image-generated/#1.pdf}}

\newcommand{\vuoImage}[2]{
	\includegraphics[scale=#1]{PROJECT_SOURCE_DIR/documentation/image/#2}
}

\newlength{\vuoImageWidth}
\newcommand{\vuoScreenshot}[1]{
	\settowidth{\vuoImageWidth}{\includegraphics[scale=0.5]{PROJECT_SOURCE_DIR/documentation/image/#1}}
	\setlength{\vuoImageWidth}{\minof{\vuoImageWidth}{\textwidth}}
	\includegraphics[scale=0.5,width=\vuoImageWidth]{PROJECT_SOURCE_DIR/documentation/image/#1.png}
}
\newcommand{\vuoCompositionImage}[1]{
	\settowidth{\vuoImageWidth}{\includegraphics[scale=0.75]{../image-generated/#1}}
	\setlength{\vuoImageWidth}{\minof{\vuoImageWidth}{\textwidth}}
	\includegraphics[scale=0.75,width=\vuoImageWidth]{../image-generated/#1.pdf}
}

\input{VuoManualColors.tex}
\AtBeginDocument{\color{base01}}
\pagecolor{base3}

\hypersetup{linkcolor=blue, citecolor=blue, urlcolor=blue}

\usepackage{listings}
	\lstdefinelanguage{dot}{
		morekeywords={},
		morekeywords=[2]{digraph,type,label},
		sensitive=false,
		morecomment=[s]{/*}{*/},
		morestring=[b]",
		moredelim=[s][\color{orange}\bfseries]{_}{=},
		otherkeywords={->,:,;},
	}
	\lstdefinelanguage[vuo]{c}{
		morekeywords={const,char,unsigned,int,void,false,true,bool,if,return,sizeof},
		morekeywords=[2]{
			VuoModuleDetails,
			nodeEvent,nodeInstanceInit,nodeInstanceCallbackStart,nodeInstanceEvent,nodeInstanceCallbackStop,nodeInstanceFini,
			VuoInstanceData,VuoInputData,VuoInputConductor,VuoOutputData,VuoOutputConductor,VuoOutputTrigger
		},
		sensitive=true,
		morecomment=[s]{/*}{*/},
		morestring=[b]",
		otherkeywords={*,=,+,;},
	}
	\lstset{
		defaultdialect=[vuo]c,
		basicstyle=\color{base00}\footnotesize\ttfamily,
		numbers=left,
		numberfirstline=false,
		numberstyle=\color{base1}\tiny,
		numbersep=5pt,
		tabsize=4,
		extendedchars=true,
		breaklines=true,
		keywordstyle=[1]\color{base03}\bfseries,
		keywordstyle=[2]\color{orange}\bfseries,
		frame=b,
		columns=fullflexible,
		stringstyle=\color{base1},
		commentstyle=\color{green},
		showspaces=false,
		showtabs=false,
		xleftmargin=0mm,
		framexleftmargin=0mm,
		framexrightmargin=0mm,
		framexbottommargin=0mm,
		showstringspaces=false,
		abovecaptionskip=5mm,
	}
	\DeclareFontShape{OT1}{cmtt}{bx}{n}{<5><6><7><8><9><10><10.95><12><14.4><17.28><20.74><24.88>cmttb10}{}

\usepackage{caption}
	\DeclareCaptionFormat{listing}{\colorbox{base2}{\parbox{12.8cm}{#1#2#3}}}
	\captionsetup[lstlisting]{format=listing, singlelinecheck=false, margin=0pt, font={bf,footnotesize}}

\setmainfont[Path=PROJECT_SOURCE_DIR/renderer/font/, BoldFont=PTS75F.ttf, ItalicFont=PTS56F.ttf, BoldItalicFont=PTS76F.ttf]{PTS55F.ttf}
\setmonofont{Monaco}

\usepackage{fontspec}
\newfontfamily\lucidaGrande{Lucida Grande}
\usepackage{newunicodechar}
\newunicodechar{⌘}{{\lucidaGrande ⌘}}
\newunicodechar{⌃}{{\lucidaGrande ⌃}}
\newunicodechar{⇧}{{\lucidaGrande ⇧}}
\newunicodechar{⌥}{{\lucidaGrande ⌥}}
\newunicodechar{⌫}{{\lucidaGrande ⌫}}
\newunicodechar{↵}{{\lucidaGrande ↵}}
\newunicodechar{⎋}{{\lucidaGrande ⎋}}
\newunicodechar{↑}{{\lucidaGrande ↑}}
\newunicodechar{↓}{{\lucidaGrande ↓}}
\newunicodechar{←}{{\lucidaGrande ←}}
\newunicodechar{→}{{\lucidaGrande →}}

\usepackage{menukeys}
	\renewmenumacro{\menu}[>]{roundedmenus}
	\renewmenumacro{\keys}[+]{roundedkeys}
	\renewmenumacro{\directory}[/]{pathswithfolder}
	\changemenucolor{gray}{bg}{named}{base2}
	\changemenucolor{gray}{br}{named}{base1}
	\changemenucolor{gray}{txt}{named}{base01}

\providecommand{\tightlist}{\setlength{\itemsep}{0pt}\setlength{\parskip}{0pt}}
